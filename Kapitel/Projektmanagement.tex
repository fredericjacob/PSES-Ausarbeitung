\section{Projektmanagement}
\label{sec:projektmanagement}
Das Projektseminar Echtzeitsysteme wird mit einer Gruppe von f\"unf Personen aus zum Teil unterschiedlichen Studienf\"achern bearbeitet. Dies stellt einen nicht unerheblichen organisatorischen Aufwand dar. Hierzu muss direkt zu Beginn des Projekts eine Grundstruktur geschaffen werden, auf welcher es sich bis zum Ende gut arbeiten l\"asst. Ganz wichtig f\"ur eine gute Zusammenarbeit in der Gruppe ist die Art der Kommunikation.

\subsection{Planung und Kommunikation}
F\"ur eine bessere Planung hat sich das Team f\"ur w\"ochentliche Meetings entschieden. Diese finden neben den zweiw\"ochentlichen Treffen mit dem Betreuer statt. Diese Meetings haben den Zweck \"uber Fortschritt der einzelnen Arbeitspakete zu sprechen, eventuelle Probleme anzusprechen und das weitere Vorgehen als Gruppe festzulegen. Das Besprochene wird auf der Plattform Trello auf einer virtuellen Pinnwand festgehalten und ist hier f\"ur jedes Teammitglied einsehbar. Man kann die Arbeitspakete zuweisen und in verschiedenen Gruppen sortieren. Somit hat man eine gute \"ubersicht \"uber den Stand des Projekts. Zus\"atzlich gibt es eine aktive Kommunikation \"uber eine eingerichtete WhatsApp-Gruppe. Hier k\"onnen Fragen beantwortet und auf Fehler schnell reagiert werden. Mit diesen Hilfsmitteln ist es uns m\"oglich jederzeit miteinander in Kontakt zu treten und den Projektstand im Blick zu haben. Somit ist jedes Teammitglied in der Lage individuell am Projekt zu arbeiten.

\subsection{Verteilte Versionsverwaltung}
Wie im vorherigem Abschnitt beschrieben, soll es jedem Mitglied m\"oglich sein, jederzeit und von jedem Ort an seiner Aufgabe zu arbeiten. Dazu ist es bei einem solchen Projekt von grossem Vorteil, wenn man eine verteilte Versionsverwaltung einsetzt. Wir haben uns in dem Fall f\"ur Git entschieden, da es alle Funktionalit\"at daf\"ur mitbringt und online, in Verbindung mit Github, kostenlos zur Verf\"ugung steht. Dadurch entstehen gleich zwei grosse Vorteile. Einerseits kann man von unterschiedlichen Systemen gleichzeitig an der gleichen Datei arbeiten und die vollbrachten Arbeiten am Ende zusammenf\"uhren. Andererseits ist eine Versionierung der Dateien intern schon vorhanden, da Git mit den Differenzen der Dateien arbeitet. Somit ist es hinf\"allig eine Datei in mehreren Version abzuspeichern. Dies spart Speicherplatz und macht die Ordnerstruktur leserlicher. Gerade bei Projekten mit textbasierter Programmiersprache ist dies eine Erleichterung.