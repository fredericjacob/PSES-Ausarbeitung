\section{Projektmanagement}
\label{sec:projektmanagement}
Das Projektseminar Echtzeitsysteme wird mit einer Gruppe von f\"unf Personen aus zum Teil unterschiedlichen Studienf\"achern bearbeitet. Dies stellt einen nicht unerheblichen organisatorischen Aufwand dar. Hierzu musste direkt zu Beginn des Projekts eine Grundstruktur geschaffen werden, auf welcher es sich bis zum Ende gut arbeiten l\"asst. Ganz wichtig f\"ur eine gute Zusammenarbeit in der Gruppe ist die Art der Kommunikation.

\subsection{Planung und Kommunikation}
F\"ur eine bessere Planung hat sich das Team f\"ur w\"ochentliche Meetings entschieden. Diese fanden neben den zweiw\"ochentlichen Treffen mit dem Betreuer jeden Freitag statt. Diese Meetings hatten den Zweck, \"uber Fortschritt der einzelnen Arbeitspakete zu sprechen, eventuelle Probleme anzusprechen und das weitere Vorgehen als Gruppe festzulegen. Zugeteilte Aufgaben wurden auf der Plattform Trello auf einer virtuellen Pinnwand festgehalten und organisiert. Diese wurden einzelnen Gruppenmitgliedern zugeteilt und nach dem Kanban-Prinzip in verschiedenen Stadien (z.B. TODO, Review, Test) organisiert. Somit war es zu jedem Zeitpunkt m\"oglich, sich einen \"Uberblick \"uber anstehende Aufgaben und den aktuellen Stand des Projekts zu verschaffen.

Zus\"atzlich wurde eine WhatsApp-Gruppe f\"ur kurzfristige Absprachen und Kommunikation eingerichtet. Diese wurde verwendet, um Fragen zu beantworten und auf Fehler schnell reagieren zu k\"onnen sowie um zus\"atzliche Termine neben den regelm\"a\ss igen Meetings auszumachen. Mit diesen Hilfsmitteln war es uns m\"oglich, jederzeit miteinander in Kontakt zu treten und den Projektstand im Blick zu haben. Somit war jedes Teammitglied in der Lage, individuell an den ihm zugeteilten Aufgaben zu arbeiten, ohne den Blick f\"ur das Gesamte zu verlieren.

\subsection{Verteilte Versionsverwaltung}
%TODO: der Abschnitt sollte noch mal überarbeitet werden... klingt teilweise zu basic und trifft m.Mn. nach noch nicht ganz den Punkt (Frederic)
Wie im vorherigem Abschnitt beschrieben, sollte es jedem Mitglied m\"oglich sein, jederzeit und von jedem Ort an seiner Aufgabe zu arbeiten. Dazu ist es bei einem solchen Projekt von gro\ss em Vorteil, wenn man eine verteilte Versionsverwaltung einsetzt. Wir haben uns in dem Fall f\"ur Git entschieden, da es alle Funktionalit\"at daf\"ur mitbringt und online, in Verbindung mit Github, kostenlos zur Verf\"ugung steht. Dadurch entstanden gleich zwei gro\ss e Vorteile. Einerseits kann man von unterschiedlichen Systemen gleichzeitig an der gleichen Datei arbeiten und die vollbrachten Arbeiten am Ende zusammenf\"uhren. Andererseits ist eine Versionierung der Dateien intern schon vorhanden, da Git mit den Differenzen der Dateien arbeitet. Somit ist es hinf\"allig eine Datei in mehreren Versionen abzuspeichern. Dies spart Speicherplatz und macht die Ordnerstruktur leserlicher. Gerade bei Projekten mit textbasierter Programmiersprache ist dies eine Erleichterung.
%TODO: welche Programmiersprache ist denn nicht textbasiert? (Frederic)