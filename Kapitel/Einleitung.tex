\section{Einleitung}
Das Projektseminar Echtzeitsysteme bietet Studenten die M\"oglichkeit echtzeitkritische Software mithilfe eines Roboterautos zu entwickeln. Dabei standen im Wintersemester 2018/2019 autonome Fahrassistenzsysteme im Vordergrund. In diesem Kapitel soll die Aufgabenstellung und die individuelle Themenwahl des Teams erl\"autert werden und ein \"Uberblick \"uber das gesamte Projekt geschaffen werden.

\subsection{Aufgabenstellung}
Die Aufgabenstellung des Projektseminars Echtzeitsysteme 2018/2019 setzte sich aus zwei Teilen zusammen. \\\\
Der erste, verpflichtende Teil stellte die Aufgabe dar, das Roboterauto selbstst\"andig eine Runde in einem vorgegebenen Rundkurs fahren zu lassen. Dabei wurden den Gruppen sowohl eine Kinect mit Farb- und Tiefenkamera, als auch eine Weitwinkel-Webcam zur Verf\"ugung gestellt. Es wurden Fahrbahnmarkierungen mit grellen Farben benutzt, um die Bildverarbeitung zu vereinfachen. Wie das Team diese erste Aufgabe dann konkret l\"osen sollte war freigestellt. \\\\
Der zweite, vom Team selbst gew\"ahlte Teil bestand darin, eine Schildererkennung zu implementieren, welche w\"ahrend des Fahrens durch den Rundkurs Stra\ss{}enschilder erkennen kann.

\subsection{\"Uberblick}
Das n\"achste Kapitel befasst sich damit, welche organisatorischen Ma\ss{}nahmen das Team ergriffen hat, um geordnet zu arbeiten. Kapitel 3 soll zun\"achst den Aufbau des Projekts darstellen und verdeutlichen, wie die Informationsstr\"ome innerhalb und zwischen den einzelnen Komponenten flie\ss{}en. Dabei stellt das Software-Framework Robot Operating System\cite{ROS} (kurz ROS) eine entscheidende Rolle dar. In Kapitel 4 werden die Filtermethoden und Kameraeinstellungen erl\"autert, welche aus dem rohen Kamerabild wichtige Informationen extrahieren. Darauf aufbauend wird in Kapitel 5 das Fahrzeugmodell vorgestellt und die Planung der Fahrtrajektorie mithilfe der gesammelten Informationen dargestellt. Das sechste Kapitel erkl\"art, welche M\"oglichkeiten genutzt wurden, um bei der Entwicklung unterschiedlicher Funktionalit\"aten auszuhelfen. Kapitel 7 besch\"aftigt sich mit der Schildererkennung mittels eines neuronalen Netzwerkes. Zuletzt wird in Kapitel 8 ein Fazit gezogen und ein Ausblick pr\"asentiert.