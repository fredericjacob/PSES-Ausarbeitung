\section{Einleitung}
In diesem Kapitel soll die Aufgabenstellung und die individuelle Themenwahl des Teams erl\"autert werden und Unterthemen in Kapitel eingegliedert werden.


\subsection{Aufgabenstellung}
Die Aufgabenstellung des Projektseminars Echtzeitsysteme 2018/2019 setzte sich aus zwei Teilen zusammen. \\\\
Der erste, vorgegebene Teil stellte die Aufgabe dar, das Roboterauto selbstst\"andig eine Runde in einem vorgegebenen Rundkurs fahren zu lassen. Dabei wurden den Gruppen sowohl eine Kinect mit Farb- und Tiefenkamera, als auch eine Weitwinkel-Webcam zur Verf\"ugung gestellt. Es wurden Fahrbahnmarkierungen mit grellen Farben benutzt, um die Bildverarbeitung zu vereinfachen. Wie wir diese erste Aufgabe dann konkret l\"osen sollten war freigestellt. \\\\
Der zweite, selbst gew\"ahlte Teil bestand darin, eine Schilderkennung zu implementieren, welche w\"ahrend des Fahrens durch den Rundkurs Stra\ss{}enschilder erkennen kann.

\subsection{Gliederung}
Kapitel 2 soll zun\"achst den Aufbau des Projekts darstellen und verdeutlichen, wie die Informationsstr\"ome innerhalb und zwischen den einzelnen Komponenten flie\ss{}en. Dabei stellt das Software-Framework Robot Operating System\footnote{http://www.ros.org/} (kurz ROS) eine entscheidende Rolle dar. In Kapitel 3 werden die Filtermethoden und Kameraeinstellungen erl\"autert, welche aus dem rohen Kamerabild wichtige Informationen extrahieren. Darauf aufbauend wird in Kapitel 4 der ...TODO