\section{Bildverarbeitung}\label{sec:linien}

Als haupts\"achliche Quelle f\"ur die Erfassung und Verarbeitung der Umwelt des Fahrzeugs haben wir das Bild
einer Webcam verwendet. Eine Kinect-Kamera mit Tiefenbild war zwar bereits auf dem Fahrzeug vorinstalliert,
jedoch beschlossen wir aus folgenden Gr\"unden, eine eigene Kamera zu verwenden:
\begin{itemize}
	\item Der flache Blickwinkel der Kinect erlaubte es nicht, den Boden unmittelbar vor dem Fahrzeug
	zu sehen. Die von uns verwendete Webcam dagegen konnte frei angebracht und der Blickwinkel
	(insbesondere dessen Steilheit) nach Belieben variiert werden.
	\item Die OpenCV-Bibliothek bot uns die M\"oglichkeit, den Webcam-Stream direkt einzulesen und
	daran Voreinstellungen vorzunehmen. Bei der Kinect w\"are hingegen eine Vorkonvertierung n\"otig
	geworden.
	\item F\"ur die von uns zu realisierenden Aufgaben (Fahrbahn- und Schildererkennung) war die
	zus\"atzliche Tiefenbild-Funktionalit\"at der Kinect wenig bis nicht relevant.
\end{itemize}
\textbf{TODO}\\
-

\subsection{Kalibrierung der Kamera}

- Beschreibung der ausgemessenen Dimensionen\\

\subsubsection{Umrechnung von Koordinaten}

- wie werden Ma\ss e f\"ur die Umrechnung verwendet?

\subsection{Einlesen des Kamera-Bilds}

- Kamera-Aufnahme-Setup \"uber OpenCV (und rqt\_reconfigure)\\
- Beschreibung der Node / des Wrappers\\
